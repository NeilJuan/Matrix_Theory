\documentclass[12pt]{article}
\usepackage{color}
\usepackage{graphics}
\usepackage{graphicx}
\usepackage{geometry}
\geometry{left=1.5cm,right=1.5cm,top=2.5cm,bottom=2.5cm}
\usepackage{latexsym,bm}
\usepackage{caption}
\usepackage{amsmath}
\usepackage{amsmath}
\usepackage{amssymb}%花体字母加粗
\usepackage{mathrsfs}%花体字母
\usepackage{CJK}
\usepackage{fancyhdr}
\pagestyle{fancy}
\lhead{} %左上头脚注
\chead{} %中间上头脚注
\rhead{\bfseries University of Chinese Academy of Science}%右边上头 
\lfoot{TianyangZhang} %左下头
\cfoot{Note of Matrix Analysis}%中下头
\rfoot{\thepage} %右下头


\begin{document}
\begin{CJK*}{UTF8}{gbsn}

%Title-------------------------------------------
\begin{titlepage}
	
\title{\protect \includegraphics[width=0.5in]{/Users/genius/Desktop/Pictures/UCAS.png} \Huge \textbf{矩阵论笔记3}}

\author{张天阳\thanks{E-mail:zhangtianyang17@mails.ucas.edu.cn}\\
University of Chinese Academy of Science}
\maketitle
%end title----------------------------------------
\end{titlepage}



%目录------------------------------------------
\tableofcontents

\pagebreak
%end 目录------------------------------------------

%-----------------正文-----------
\part{矩阵分解}
\section{Gauss消去法:}
\begin{equation*}
	A=\begin{bmatrix}
	a_{11}&a_{12} &\cdots &a_{13} \\
	a_{21}&a_{22} &\cdots &a_{23} \\
	\vdots  &\vdots &\ddots &\vdots \\
	a_{n1}&a_{n2}&\cdots &a_{n3}
\end{bmatrix}
\end{equation*}
首先行变换左乘,列变换右乘\\
每次高斯消去都是左乘第k次变换的矩阵$L_k$:
\begin{equation}
	L_k=I-\bm l_k\bm e^T
\end{equation}
\begin{equation}
	\bm l_k=[0_1,0_2,0_3,...,0_{k},c_{k+1},c_{k+2},c_{k+3},...,c{n}]^T,~~0_k~means~kth~0
\end{equation}
因为每次高斯消去法都是将第 k 列消去,所以就是减去一个第 k 行从 $a_{kk}$ 开始往下都是$\frac{a_{k(k+i)}}{a_{kk}}$乘以第k行再减去自己,于是相当于乘以:
\begin{equation}
	\bm l_k\bm e^T=\begin{bmatrix}
	0&\cdots & 0 &\cdots & 0\\
	\vdots &\cdots & \vdots &\cdots & \vdots\\
	0&\cdots & 0&\cdots & 0\\
	\vdots &\cdots &0_{k,k}&\cdots & \vdots\\
	0 &\cdots &\frac{a_{k,(k+1)}}{a_{k,k}}&\cdots & 0\\
	0 &\cdots &\vdots &\cdots & 0\\
	0 &\cdots &\frac{a_{k,(n)}}{a_{k,k}}&\cdots & 0\\
\end{bmatrix}
\end{equation}
\begin{equation}
	L=L_1L_2...L_n=I-\bm l_1\bm e^T-\bm l_2\bm e^T-...-\bm l_{n-1}\bm e^T
\end{equation}
则上三角矩阵$LA$可得。
\section{三角分解:}
\begin{equation}
	A=LU
\end{equation}
先求出Gauss消去法的$L$。\\
矩阵分解中$L$是Gauss中的$L^{-1}$
\begin{equation}
	L=L_1L_2...L_n=I+\bm l_1\bm e^T+\bm l_2\bm e^T+...+\bm l_{n-1}\bm e^T
\end{equation}
而$U$就是Gauss消去法所得到的上三角矩阵。
\subsection{Doolittle分解}
即将矩阵分解为$LDU$其中$U$为对角值为1的矩阵,而$D$则是对角矩阵,其实就是讲矩阵分解为$LU$之后将$U$逐行提取公因式,写作$D$。\\
而$LU$则称为Doolittle分解。
\subsection{Cholesky分解}
Choleshy分解是将\textcolor{blue}{是对称正定矩阵$A$}分解为
\begin{equation}
	A=G^TG
\end{equation}
其是现将矩阵分解为:$LDU$形式:\\
可得\begin{equation}
	G=L\sqrt{D}
\end{equation}
即$D$的每个对角线的元素($D$为对角矩阵)都开跟号

\section{分块矩阵求逆}
\subsection{行变换解析}
\begin{equation}
	\begin{bmatrix}
	A& B \\
	C& D 
\end{bmatrix}=
\begin{bmatrix}
	A& B& \\
	-CA^{-1}A+C& -CA^{-1}B+D& 
\end{bmatrix},~~Gauss~Method
\end{equation}
相当于左乘(行变换),注:分块矩阵乘法是原来在左边的乘完还在左边。
\begin{align*}
	X=\begin{bmatrix}
	1& &&  &&&\\
	& 1&& &&&\\
	& &1& &&&\\
	& &&\ddots &&&\\
	0&\cdots &x_{ij}&\cdots &1&\cdots &0\\
	& && &&\ddots &\\
	& && &&&1\\
	\end{bmatrix},only~X_{ij}=x_{ij}\\
	\textcolor{blue}{X=I+x_{ij}e_ie_j^T}
\end{align*}
这样的变换矩阵$X$,左乘$A$,$XA$相当于将$A$的第~i~行加上第~j~列乘以$x_{ij}$\\

解释上面为什么是第一行左乘$CA^{-1}$在加到第二行,因为高斯消去法需要让第二行第一个元素为$O$所以要将第一行乘以(左乘是因为行变换是左乘,$CA^{-1}$是因为$A^{-1}C$的话消不掉$A$会变为$A^{-1}CA$)。
\textbf{\textcolor{blue}{该行都同时乘以A的话是左乘,而该列都同时乘以A的话是右乘。}}
\subsection{分块矩阵求逆}
对下面分块矩阵求逆:应该先拓展,然后将其行变换为单位阵然后拓展阵自然变成它的逆。
\begin{equation*}
\begin{bmatrix}
	A& B \\
	C& D 
\end{bmatrix}\Rightarrow 
\begin{bmatrix}
	A& B &I&O\\
	C& D &O&I
\end{bmatrix}
\end{equation*}
下面的式子第二行应该归一化,因为不然化三角时会让式子很长。
\begin{align*}
	\begin{bmatrix}
	A& B &I&O\\
	C& D &O&I
\end{bmatrix}=\begin{bmatrix}
	A& B& I&O\\
	-CA^{-1}A+C& -CA^{-1}B+D& O-CA^{-1}B&I-CA^{-1}B
\end{bmatrix}\\
=\begin{bmatrix}
	A& B & I &O\\
	O& I& -(-CA^{-1}B+D)^{-1}CA^{-1}B& (-CA^{-1}B+D)^{-1}(I-CA^{-1}B)
\end{bmatrix}\\
=\begin{bmatrix}
	A& O & I+B^{-1}(-CA^{-1}B+D)^{-1}CA^{-1}B &-B^{-1}(-CA^{-1}B+D)^{-1}\\
	O& I& -(-CA^{-1}B+D)^{-1}CA^{-1}B& (-CA^{-1}B+D)^{-1}(I-CA^{-1}B)
\end{bmatrix}
\end{align*}
\begin{equation*}
	=\begin{bmatrix}
	I& O & A^{-1}[I+B^{-1}(D-CA^{-1}B)^{-1}CA^{-1}B] &-A^{-1}[B^{-1}(D-CA^{-1}B)^{-1}]\\
	O& I& -(D-CA^{-1}B)^{-1}CA^{-1}B& (D-CA^{-1}B)^{-1}(I-CA^{-1}B)
\end{bmatrix}
\end{equation*}
所以其逆为:
\begin{equation*}
	=\begin{bmatrix}
	 A^{-1}[I+B^{-1}(D-CA^{-1}B)^{-1}CA^{-1}B] &-A^{-1}[B^{-1}(D-CA^{-1}B)^{-1}]\\
	-(D-CA^{-1}B)^{-1}CA^{-1}B& (D-CA^{-1}B)^{-1}(I-CA^{-1}B)
\end{bmatrix}
\end{equation*}

\section{Givens矩阵(旋转矩阵)对向量操作:}
\subsection{二维例子}
顺时针旋转~$\theta$角(在极坐标系中角度增加$-\theta$)后空间中的向量$\bm x$为:
因为极坐标中的$\bm x$为:
\begin{equation*}
	x_1=r\cdot cos(\varphi),~~x_2=r\cdot sin(\varphi)
\end{equation*}
\begin{equation}
	\begin{bmatrix}
	cos(\theta)&sin(\theta) \\
	-sin(\theta)& cos(\theta) \\
\end{bmatrix}\bm x=T_{12}\bm x
\end{equation}
用c表示$cos(\theta)$,s表示$sin(\theta)$

\subsection{高维时的旋转矩阵}
高纬时候其仍然一次只旋转两个维度,因为任意两个维度可以组成一个平面,然后在该平面上有上面二维的结论,故当旋转将~i~向着~j~旋转时:
\begin{align*}
	\bm x=[x_1,x_2,...,x_n]^T,~~~\bm y=T_{ij}\bm x=[y_1,y_2,...,y_n]^T\\
	\begin{cases}
		y_i=cx_i+sx_j\\
		y_j=-sx_i+sx_j\\
		y_k=x_k,~~~(if~k\neq i,j)
	\end{cases}
\end{align*}
$T_{ij}$被叫做GIVENS矩阵
\subsection{旋转定理}
1.存在有限个Givens矩阵的乘积,记作$T$可以使$T\bm x=|\bm x|\bm e_1$,~~$\bm e_i$为仅仅只有第i维为1的单位列向量。\\
2.同时也存在有限个Givens矩阵的乘积,记作$T$可以使$T\bm x=|\bm x|\bm z$,$z$为任意单位向量



\section{Householder变换(初等反射变换)}
\subsection{Householder矩阵定义与推导}
\begin{equation}
	H=I-2\bm u\bm u^T
\end{equation}
$\bm u$是$\bm x$要对称的超平面的法向量(单位向量,方向垂直于超平面),$\bm\mu$为图上与法向量方向相同的向量,推导:
\begin{figure}[h]
	\center
  \includegraphics[width=4in]{/Users/genius/Desktop/MatrixAnslysisNote/1.png}
  \caption{反射变换图}
\end{figure}
\begin{align*}
	According~to~the~defination:~~H\bm x=\bm y\\
	\bm x-\bm y =2\bm \mu\\
	\Rightarrow (I-H)\bm x=2\bm \mu\\
	Becasus~  \bm u^T\bm x=\|\bm \mu \|,~So~~\bm u\bm u^T\bm x=\bm \mu\\
	\Rightarrow H=I-2\bm u\bm u^T
\end{align*}
\subsection{Householder变换性质}
1.~~H矩阵性质:
\begin{equation}
	\begin{cases}
		H^T=H~~~~,~~(\lambda~~is~~real~number)\\
		H^TH=I,(Thus~H^T=H^{-1})~~~,~~H~is~Orthogonal~Matrix\\
		H^2=I\\
		H^{-1}=H\\
		|H|=-1
	\end{cases}
\end{equation}
2.~~对于任意$\bm x\in R^{n}$机器单位列向量$z\in R^{n}$存在Householder矩阵H,使得$Hx=|x|\bm z$,$z$为任意单位向量。\\
3.~~任意Givens矩阵(旋转矩阵)可以由两个Householder矩阵相乘得到,但是Householder矩阵不可以由若干Givens矩阵相乘得到,因为$|G|=1,|H|=-1$。

\section{矩阵的QR分解}
\subsection{定义}
矩阵的QR分解是将满秩矩阵$A$分解为$A=QR$其中$Q$为正交矩阵(酉矩阵$Q^TQ=I$),而$R$则为上三角矩阵。
\subsection{方法一:Schmidt正交化方法}
因为$A$为满秩矩阵所以$A$的所有列是线性无关的,所以其所有列可以正交化得到一组正交基,将正交化的列放在一起便成了正交矩阵$B$。具体过程如下:
\begin{enumerate}
	\item 对于矩阵$A=(\bm a_1,\bm a_2,...,\bm a_n)$作正交化:Schmidt正交化过程可以想象一个下三角矩阵,其每个下标是$k_{ij}$而其第n行(每一行)都是$\bm b_n$对于$\bm a_n$的线性组合,而每行的$\bm b_k$表示列的下标。
	\begin{align*}
			\mbox{C的行为下面的列、}&\mbox{第一行、}\mbox{第二行...}\mbox{...第n
		行}\\
		\bm a_1&=\bm b_1\\
		\bm a_2&=k_{21}\bm b_1+\bm b_2\\
		\bm a_3&=k_{31}\bm b_1+k_{32}\bm b_2+\bm b_3\\
		...\\
		\bm a_n&=k_{n1}\bm b_1+k_{n2}\bm b_2+....+\bm b_n
	\end{align*}
	\textcolor{blue}{几何意义想象一下,第k行为$\bm a_k$在$\bm b_1,...,\bm b_k$的空间内的坐标,其中$k_{ij}$为第j个坐标的值为:$\bm a_k$在$\bm b_j$上的投影乘以单位化的$\bm b_{j}$即:}
	\item 其中$\bm a_k$是已知要被分解的矩阵$A$的第k列,而$k_{ij}$为:
	\begin{equation}
		k_{ij}=\frac{(\bm a_i,\bm b_j)}{(\bm b_j,\bm b_j)}
	\end{equation}
	所以$k_{ii}=1$所以对角线上的k为1。于是可以得到$b_i$的公式:

		\begin{align*}
		\bm b_1&=\bm a_1\\
		\bm b_2&=\bm a_2-k_{21}\bm b_1\\
		\bm b_3&=\bm a_3-k_{31}\bm b_1-k_{32}\bm b_2\\
		...\\ 
		\bm b_n&=\bm a_n-k_{n1}\bm b_1-k_{n2}\bm b_2-....-k_{nn-1}\bm b_{n-1}
	\end{align*}
	所以$A=BC$,$C$为上面所述。\\
	我们求出所有$\bm b_i$之后我们将其对角化之后便是$Q$
	而对角化需要乘以对角矩阵:\\
	$diag(1/\|\bm b_1\|,1/\|\bm b_2\|,...,1/\|\bm b_n\|)$\\
	矩阵$diag(|\bm b_1\|,\|\bm b_2\|,...,\|\bm b_n\|)C$即为$R$
	\item 综上所述可得$A=QR$
\end{enumerate}
\subsection{方法二:Givens矩阵变换得到QR分解}
Givens变换$T_1$可以使得矩阵$A=(\bm a_1,\bm a_2,...,\bm a_n)$的第一列变为~$[\|\bm a_1\|,0,0,...,0]^T$,于是将矩阵A乘以$T_1$可以得到:($T_1$变换时最接近$A$,即将$A$用许多Givens变换变为上三角矩阵(R)时$T_1$为最接近$A$的那个左乘矩阵)
\begin{equation}
	T_1A=\begin{bmatrix}
	\|\bm a_1\|& *&..&* \\
	0& & \\
	\vdots &&A_1\\
	0& 
\end{bmatrix}_{n\times n}
\end{equation}
再对$A_1$的第一列进行Givens变换$T_2A_1$
\begin{equation}
	T_2A_1=\begin{bmatrix}
	\|\bm a^{(1)}_1\|& *&..&* \\
	0& & \\
	\vdots &&A_2\\
	0& 
\end{bmatrix}_{(n-1)\times (n-1)}
\end{equation}
以此类推:
\begin{equation}
	T=
	\begin{bmatrix}
	I_{n-2}& O \\
	O& T_{n-1} 
\end{bmatrix}\begin{bmatrix}
	I_{n-3}& O \\
	O& T_{n-2} 
\end{bmatrix}...\begin{bmatrix}
	1 &O \\
	O& T_2 
\end{bmatrix}T_1
\end{equation}
\begin{equation}
	TA=R
\end{equation}
于是:
\begin{equation}
	A=T^{-1}R=QR
\end{equation}
\subsection{方法三:利用Householder变换得到QR分解}
思路与用Givens矩阵得到QR分解一样也是先用变换矩阵T得到上三角矩阵R\\
同样先对$A=(\bm a_1,\bm a_2,...,\bm a_n)$的第一列找到变换矩阵$H_1$使得$\bm a_1$变为:$\|\bm a_1\|\bm e_1$,然后以此类推。
\begin{equation}
	T=
	\begin{bmatrix}
	I_{n-2}& O \\
	O& H_{n-1} 
\end{bmatrix}\begin{bmatrix}
	I_{n-3}& O \\
	O& H_{n-2} 
\end{bmatrix}...\begin{bmatrix}
	1 &O \\
	O& T_2 
\end{bmatrix}T_1
\end{equation}
\begin{equation}
	TA=R
\end{equation}
于是:
\begin{equation}
	A=T^{-1}R=T^TR=QR
\end{equation}
对于$H$的求解方法:$H=I-2\bm u\bm u^T$而对于$\bm u$的求解:
\begin{align*}
	\bm \mu= \bm a_1-\|\bm a_1\|\bm e_1\\
	\bm u=\frac{\bm \mu}{\|\mu\|}
\end{align*}
\pagebreak
\section{矩阵的满秩分解:FG分解}
用于分解一些不是方阵(非方阵)的矩阵。\\
分解方法:
\begin{enumerate}
	\item 用高斯消去法将待分解的矩阵转换为Hermite标准型(不是Hermite矩阵)\\
	Hermite标准型:每一行第一个非零元素为1,前r行为非零行。
	\item 取A的每个Hermite(标准型)对应的首1列(行的第一个非零的元素也就是1对应的列)组成F,而Hermite标准型的所有非零行构成G。
\end{enumerate}
\textcolor{blue}{满秩分解重要性质:$A_r^{m\times n}=F_r^{m\times r}G_r^{r\times n}$}

\section{奇异值分解}
\subsection{前序知识与回顾}
\subsubsection{谱分解}
如果\textbf{\textcolor{blue}{$A$是实对称矩阵}}那么A满足:
\begin{equation}
	Q^TAQ=diag(\lambda_1,\lambda_2,...,\lambda_n)\mbox{,其中Q为正交矩阵}
\end{equation}
如果A不是实数对称矩阵,但是$A$仍然非奇异,那么$A$可以分解为:
\begin{equation}
	P^TAQ=diag(\lambda_1,\lambda_2,...,\lambda_n)\mbox{,其中P与Q为正交矩阵}
\end{equation}
而奇异值分解为无任何条件的矩阵$A$可以分解为:
\begin{equation}
	U^HAV=\begin{bmatrix}
	\Sigma_{r\times r} &O  \\
	O&  O\\
\end{bmatrix}
\end{equation}
即:$V$是左乘$A$的所以对于运算时用$A^TA$还是$AA^T$,可以先求出$V$或者$U$
\begin{equation}
	A=U\begin{bmatrix}
	\Sigma_{r\times r} &O  \\
	O&  O\\
\end{bmatrix}V^H
\end{equation}

\subsection{奇异值分解}
\subsubsection{思路与引理:其与前序的方法不同也不要求$A$是非奇异}
\begin{enumerate}
	\item 由于$A$虽然不是实对称矩阵,$A$为$\forall A\in \bm C^{m\times n} $,但是$A^TA$是Hermite矩阵(对称矩阵)所以其是半正定的。
	\item 且$rank(A^TA)=rank(A)$
	\item $A^TA=O$的冲要条件是$A=O$
	\item 对$A^TA$进行谱分解可以得到:$V^TA^TAV=diag(\lambda_1,\lambda_2,...,\lambda_n)\mbox{,其中V为正交(酉)矩阵}$其中\textcolor{blue}{其中$\lambda_1,\lambda_2,...,\lambda_n$为矩阵$A^TA$的特征值,而$\sigma_i=\sqrt{\lambda_i},~(i=1,2,...,n)$被称为其奇异值。}\\
对于V取其每一列$\bm v_i$可以发现:
	\begin{equation}
		A^TA\bm v_i=\bm v_i\lambda_i
	\end{equation}
	所以求出每个$\lambda_i$对应的特征向量$\bm \xi_i$,(零特征值的也求,重根也求,求n个嘛)(每个特征值只求一个即可),可得$V=(\bm\xi_1,\bm\xi_2,...,\bm\xi_n)$,
\end{enumerate}
\subsubsection{奇异值分解证明}
\begin{enumerate}
	\item 上一小节得到的矩阵$B=A^TA$的特征值对角矩阵$diag(\lambda_1,\lambda_2,...,\lambda_n)$其中有r个非零项,所以将其写为:
	\begin{equation}
		diag(\lambda_1,\lambda_2,...,\lambda_n)=\begin{bmatrix}
	\Sigma^2_{r\times r} &O  \\
	O&  O\\
\end{bmatrix}
	\end{equation}
	\item 求出每个$\lambda_i$对应的特征向量,(零特征值的也求,重根也求,求n个嘛)(每个特征值只求一个即可),可得$(\bm\xi_1,\bm\xi_2,...,\bm\xi_n)$,由于谱分解的定律我们可以得到:
	\begin{eqnarray*}
		V^HA^TAV=\begin{bmatrix}
	\Sigma^2_{r\times r} &O  \\
	O&  O\\
\end{bmatrix}\\
V=(\bm\xi_1,\bm\xi_2,...,\bm\xi_n)
	\end{eqnarray*}
	取$V_{n\times n}$的前r列构成矩阵$V_1$可以得到:$V=[{V_1}_{(n\times r)}|{V_2}_{(n\times n-r)}]$\\
	可知:\begin{equation}
		V_1^HA^TAV_1=\Sigma^2_{(r\times r)}
	\end{equation}
	\item 可以将上式写成:\begin{align*}
		A^TA=V_1\Sigma^2V_1^H,~\mbox{因为V正交}\\
		\Rightarrow (AV_1\Sigma^{-1})^H(AV_1\Sigma^{-1})=I\\
	\end{align*}
	所以:令~$U_1=AV_1\Sigma^{-1}$可得
	\begin{equation}
		\begin{cases}
			U_1^TU_1=I,~\mbox{得到$U_1$为正交矩阵}\\
			U_1^HAV_1=\Sigma
		\end{cases}
	\end{equation}
	让$U=[U_1|U_2]$其中$U_2$为随意列只要保证$U$正交即可:
	因为:
	\begin{align*}
		U^HAV=&[U_1|U_2]^HA[V_1|V_2]\\
		=&[U_1A|U_2A]^H[V_1|V_2]\\
		=&\begin{bmatrix}
	U_1^HAV_1&U_1^HAV_2  \\
	U_2^HAV_1&U_2^HAV_2  \\
\end{bmatrix}
	\end{align*}
\end{enumerate}
又因为
\begin{align*}
	AV_1=U_1\Sigma \Rightarrow U_2^HAV_1=U_2^HU_1\Sigma=O,\mbox{($U_2,U_1$正交)}\\
	AV_2=O\\
	U_1^HAV_1=\Sigma\\
	\Rightarrow U^HAV=\begin{bmatrix}
	\Sigma &O  \\
	O&  O
\end{bmatrix}
\end{align*}
\subsubsection{用$A^TA$奇异值分解使用方法}
\textbf{\textcolor{blue}{记住$VA^TAV$在一起,即V为n阶方阵}}
\begin{enumerate}
	\item 得到的矩阵$B=A^TA$的特征值对角矩阵$diag(\lambda_1,\lambda_2,...,\lambda_n)$其中有r个非零项,所以将其写为:
	\begin{equation}
		diag(\lambda_1,\lambda_2,...,\lambda_n)=\begin{bmatrix}
	\textcolor{blue}{\Sigma^2_{r\times r}} &O  \\
	O&  O\\
\end{bmatrix}
	\end{equation}
	\item 求出每个$\lambda_i$对应的特征向量,(零特征值的也求,重根也求,求n个嘛)(每个特征值只求一个即可),可得$(\bm\xi_1,\bm\xi_2,...,\bm\xi_n)$,由于谱分解的定律我们可以得到:
	\begin{eqnarray*}
		V^HA^TAV=\begin{bmatrix}
	\Sigma^2_{r\times r} &O  \\
	O&  O\\
\end{bmatrix}\\
V=(\bm\xi_1,\bm\xi_2,...,\bm\xi_n)
	\end{eqnarray*}
	取$V_{n\times n}$的前r列构成矩阵$V_1$可以得到:$V=[{V_1}_{(n\times r)}|{V_2}_{(n\times n-r)}]$\\
	\item 令~$U_1=AV_1\Sigma^{-1}$(用$U_1^HAV=\Sigma$记忆。)可得
	$U=[U_1|U_2]$让其中$U_2$为随意列只要保证$U$正交即可:
	\item 则:
	\begin{equation}
		U^HAV=\begin{bmatrix}
	\Sigma &O  \\
	O&  O
\end{bmatrix}
	\end{equation}
\end{enumerate}
\subsubsection{用$AA^T$奇异值分解使用方法}
\textbf{\textcolor{blue}{记住$UAA^TU$在一起,即U为m阶方阵}}
\begin{enumerate}
	\item 得到的矩阵$C=AA^T$的特征值对角矩阵$diag(\lambda_1,\lambda_2,...,\lambda_n)$其中有r个非零项,所以将其写为:
	\begin{equation}
		diag(\lambda_1,\lambda_2,...,\lambda_n)=\begin{bmatrix}
	\textcolor{blue}{\Sigma^2_{l\times l}} &O  \\
	O&  O\\
\end{bmatrix}
	\end{equation}
	\item 求出每个$\lambda_i$对应的特征向量,(零特征值的也求,重根也求,求n个嘛)(每个特征值只求一个即可),可得$(\bm\eta_1,\bm\eta_2,...,\bm\eta_n)$,由于谱分解的定律我们可以得到:
	\begin{eqnarray*}
		U^HAA^TU=\begin{bmatrix}
	\Sigma^2_{l\times l} &O  \\
	O&  O\\
\end{bmatrix}\\
U=(\bm\eta_1,\bm\eta_2,...,\bm\eta_n)
	\end{eqnarray*}
	取$U_{n\times n}$的前r列构成矩阵$U_1$可以得到:$U=[{U_1}_{(n\times l)}|{U_2}_{(n\times n-l)}]$\\
	\item 令~$V_1=(U_1^HA\Sigma^{-1})^H$可得
	$V=[V_1|V_2]$让其中$V_2$为随意列只要保证$V$正交即可:
	\item 则:
	\begin{equation}
		U^HAV=\begin{bmatrix}
	\Sigma &O  \\
	O&  O
\end{bmatrix}
	\end{equation}
\end{enumerate}
\subsection{奇异值分解的定理与应用}
\subsubsection{定理}
\begin{enumerate}
	\item $A=\sigma_1\bm u_1\bm v_1^H+\sigma_2\bm u_2\bm v_2^H+...+\sigma_n\bm u_n\bm v_n^H$
	\item $N(A)=L(\bm v_{r+1},\bm v_{r+2},...,\bm v_{r+n})$
	\item $R(A)=L(\bm u_1,\bm u_2,...,\bm u_n)$
\end{enumerate}
\subsubsection{应用}
数据复原上我们有了一些数据,就是数据矩阵上的一些点是已知$X_{ij},~(i,j)\in \mathscr{A} $的但是矩阵中存在未知的元素在里面。\\
我们要求已知点被还原时满足
\part{矩阵的特征值的重要性质:}
\large
\color{blue}
\begin{enumerate}
	\item 其中$\lambda(A)$代表$A$的特征值
	\item \begin{align*}
	\lambda(B)=1/\lambda(B^{-1}),\lambda(A^TA)=\lambda(AA^T)
\end{align*}
\item	即$A$的特征值是$\lambda$,则$A^{-1}$的特征值是$\frac{1}{\lambda}$
\item	且$A^{k}$的特征值为$\lambda^k$
\item $AA^T$的非零特征值为$A^TA$的非零特征值,$AB$与$BA$的非零特征值一样的,又因为$tr(AB)=tr(BA)$所以非零值不仅一样重数也一样。
\color{black}
\begin{align*}
\mbox{推导}\\
	A\bm x=\lambda \bm x,~~B\bm x=\beta \bm x\\
	BA\bm x=B\lambda \bm x=\lambda B\bm x=\lambda \beta \bm x\\
	AB\bm x=A\beta \bm x=\beta A\bm x=\beta\lambda  \bm x
\end{align*}
\end{enumerate}





%-----------------结束-----------
\end{CJK*}
\end{document}
